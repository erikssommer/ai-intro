\documentclass{article}
\usepackage[utf8]{inputenc}

\title{Assignment 5: Propositional and First Order Logic in Reasoning}
\author{Erik Storås Sommer, 535006}
\date{November 2022}

\begin{document}

\maketitle

\section{Models and entailment in propositional logic}

\subsection{Modelling}

Truth table\\

a)\\
\begin{displaymath}
\begin{array}{|c|c|c|c|c}

A & 
B & 
\neg A \wedge \neg B &
\neg A \wedge \neg B \Rightarrow \neg B\\ % Use & to separate the columns
\hline  % Put a horizontal line between the table header and the rest.
T & T & F & T\\
T & F & F & T\\
F & T & F & T\\
F & F & T & T\\

\end{array}
\end{displaymath}


b)\\
\begin{displaymath}
\begin{array}{|c|c|c|c|c}

A &
B & 
\neg A \wedge \neg B &
\neg A \vee \neg B \Rightarrow \neg B\\ % Use & to separate the columns
\hline  % Put a horizontal line between the table header and the rest.
T & T & F & T\\
T & F & T & T\\
F & T & T & F\\
F & F & T & T\\

\end{array}
\end{displaymath}


c)\\
\begin{displaymath}
\begin{array}{|c|c|c|c|c|c}

A & 
B & 
\neg A \wedge B & 
A \vee B & \neg A \wedge B \Rightarrow A \vee B\\ % Use & to separate the columns
\hline  % Put a horizontal line between the table header and the rest.
T & T & F & T & T\\
T & F & F & T & T\\
F & T & T & T & T\\
F & F & F & F & T\\

\end{array}
\end{displaymath}


d)\\
\begin{displaymath}
\begin{array}{|c|c|c|c|c|c}

A & 
B & 
A \Rightarrow B & 
A \Leftrightarrow B & 
(A \Rightarrow B) \Rightarrow (A \Leftrightarrow B)\\ % Use & to separate the columns
\hline  % Put a horizontal line between the table header and the rest.
T & T & T & T & T\\
T & F & F & F & T\\
F & T & T & F & F\\
F & F & T & F & T\\

\end{array}
\end{displaymath}

e)\\
\begin{displaymath}
\begin{array}{|c|c|c|c|c|c|c|c}

A & 
B & 
C &
A \Rightarrow B & 
(A \Rightarrow B) \Leftrightarrow C &
A \vee \neg B \vee C &
((A \Rightarrow B) \Leftrightarrow C) \Rightarrow (A \vee \neg B \vee C)\\ % Use & to separate the columns
\hline  % Put a horizontal line between the table header and the rest.
T & T & T & T & T & T & T\\
T & T & F & T & F & T & T\\
T & F & T & F & F & T & T\\
T & F & F & F & T & T & T\\
F & T & T & T & T & T & T\\
F & T & F & T & F & F & T\\
F & F & T & T & T & T & T\\
F & F & F & T & F & T & T\\

\end{array}
\end{displaymath}

f)\\
\begin{displaymath}
\begin{array}{|c|c|c|c|c|c}

A & 
B & 
\neg A \Rightarrow \neg B & 
A \wedge \neg B &
(\neg A \Rightarrow \neg B) \wedge (A \wedge \neg B)\\ % Use & to separate the columns
\hline  % Put a horizontal line between the table header and the rest.
T & T & T & F & F\\
T & F & T & T & T\\
F & T & F & F & F\\
F & F & T & F & F\\

\end{array}
\end{displaymath}


g)\\
\begin{displaymath}
\begin{array}{|c|c|c|c|c|c}

A & 
B & 
\neg A \Rightarrow \neg B & 
A \wedge \neg B &
(\neg A \Leftrightarrow \neg B) \wedge (A \wedge \neg B)\\ % Use & to separate the columns
\hline  % Put a horizontal line between the table header and the rest.
T & T & T & F & F\\
T & F & F & T & F\\
F & T & F & F & F\\
F & F & T & F & F\\

\end{array}
\end{displaymath}

\subsection{Trouble in the lab}

Design a control system using model checking to close tank gates whenever an 8-bit packet arrives:\\

a)\\

b)\\

c)\\

d)\\

\section{Resolution in propositional logic}

\subsection{Conjunctive Normal Form}

Convert each of the following sentences to their Conjunctive Normal Form (CNF)\\

a)\\
A \vee (B \wedge C \wedge \neg D)\\
(A \vee B) \wedge (A \vee C) \wedge (A \vee \neg D)\\

b)\\
\neg (A \Rightarrow \neg B) \wedge \neg (C \Rightarrow \neg D)\\
\neg (\neg A \vee \neg B) \wedge \neg (\neg C \vee \neg D)\\
(A \vee B ) \wedge (C \vee D)\\

c)\\
\neg ((A \Rightarrow B) \wedge (C \Rightarrow D))\\
\neg ((\neg A \vee B) \wedge (\neg C \vee D))\\
(\neg (\neg A \vee B) \wedge \neg(\neg C \vee D))\\
(A \vee \neg B) \wedge (C \vee \neg D)\\

d)\\
(A \wedge B) \vee (C \Rightarrow D)\\
(A \wedge B) \vee (\neg C \vee D)\\
NOT \: COMPLETE\\

e)\\
A \Leftrightarrow (B \Rightarrow \neg C)\\
(A \Rightarrow (B \Rightarrow \neg C)) \wedge ((B \Rightarrow \neg C) \Rightarrow A)\\
(\neg A \vee (B \Rightarrow \neg C)) \wedge (\neg (B \Rightarrow \neg C) \vee A)\\
(\neg A \vee (\neg B \vee \neg C)) \wedge (\neg (\neg B \vee \neg C) \vee A)\\
(\neg A \vee (\neg B \vee \neg C)) \wedge ((B \vee C) \vee A)\\
(\neg A \vee \neg B \vee \neg C) \wedge (B \vee C \vee A)\\

\subsection{Inference in propositional logic}

\section{Representation in First-Order Logic (FOL)}

\subsection{Predicates}
Vocabulary:\\
1. Occupation(p, o) is a predicate where person p has occupation o\\
2. Customer(p1, p2) is a predicate where person p1 is a customer of person p2 3. Boss(p1, p2) is a predicate where person p1 is a boss of person p2\\
4. Doctor, Surgeon, Lawyer, Actor are constants denoting an occupation\\
5. Emily, Joe are constants denoting people\\



a) Emily is either a surgeon or a lawyer\\
Occupation(Emily, \neg Surgeon) \wedge Occupation(Emily, \neg Lawyer)\\

b) Joe is an actor, but he also holds another job\\
Occupation(Joe, Actor) \wedge (Occupation(Joe, Doctor) \vee Occupation(Joe, Surgeon) \vee Occupation(Joe, Lawyer))\\

c) All surgeons are doctors\\
\forall x \: Occupation(x, Surgeon) \Rightarrow Occupation(x, Doctor)\\

d) Joe does not have a lawyer (i.e. he’s not a customer of any lawyer)\\
\exists x \: Customer(Joe, \neg Occupation(x, Lawyer))\\

e) Emily has a boss who is a lawyer\\
\exists x \: Boss(Emily, Occupation(x, Lawyer))\\

f) There exists a lawyer all of whose customers are doctors\\
\exists x \forall y \: Customer(Occupation(x, Doctor), Occupation(x, Lawyer))\\

g) Every surgeon has a lawyer\\
\forall x \exists y \: Customer(Occupation(x, Surgeon), Occupation(y, Doctor))\\


\subsection{Functions as predicates}

Arithmetic assertions can be written using FOL. Use the predicates (<,≤,̸=,=), the usual arithmetic operations (+, −, ×, /) as function symbols, biconditionals to create new predicates, and integer number constants to express the following statements in FOL:\\

a) Divisible(x, y): an integer number x is divisible by y if there is some integer z less than x such that x = z × y.\\

b) Even(x): a number is even if and only if it is divisible by 2.\\

c) Odd(x): a number is odd if it is not divisible by 2.\\

d) Odd(x): a number is odd if it is the result of summing 1 to an even number.\\

e) Prime(x): a number is prime if is divisible only by itself.\\

f) There is only one even prime number.\\

g) Every integer number is equal to a product of prime numbers (Hint: you can use Qki=1 pk to express a product of numbers, or use . . . to express a repeating pattern like p1, . . . , pn meaning p1, p2, p3 and on until pn).\\

\section{Resolution in FOL}

Rules: \\
• All users who like GG are known as Sone, and therefore all Sone like GG\\
• All users who like RV are known as Reveluvs, and therefore all Reveluvs like RV\\
• All users who like BP are known as Blinks, and therefore all Blinks like BP\\
• All users who identify as Revelusv will always like Ballads\\
• All users who identify as Blinks will always like Dance\\
• All users who like both Dance and Ballads will always like CH\\
• All users who like both Drama and Ballads will always like HE\\
• For all users who identify as Sone, the following holds:\\
– If they like Electro, they will always like DJH\\
– If they like Drama, they will always like SEO\\
– If they like Ballads, they will always like TAE\\

a) Generate the knowledge base by converting each rule to symbolic form.\\

1. \forall x \: GG(x) \Leftrightarrow Sone(x)\\

\: \: \: \: - CNF: (\neg GG(x) \vee Sone(x)) \wedge (\neg Sone(x) \vee GG(x))\\

2. \forall x \: RV(x) \Leftrightarrow Revelusv(x)\\

\: \: \: \: - CNF: (\neg RV(x) \vee Reveluvs(x)) \wedge (\neg Reveluvs(x) \vee RV(x))\\

3. \forall x \: BP(x) \Leftrightarrow Blinks(x)\\

\: \: \: \: - CNF: (\neg BP(x) \vee Blinks(x)) \wedge ( \neg Blinks(x) \vee BP(x))\\

4. \forall x \: Revelusv(x) \Rightarrow Ballads(x)\\

\: \: \: \: - CNF: \neg Reveluvs(x) \wedge Ballads(x)\\

5. \forall x \: Blinks(x) \Rightarrow Dance(x)\\

\: \: \: \: - CNF: \neg Blinks(x) \vee Dance(x)\\

6. \forall x \: (Dance(x) \wedge Ballads(x)) \Rightarrow CH(x)\\

\: \: \: \: - CNF: \neg Dance(x) \vee \neg Ballads(x) \vee CH(x)\\

7. \forall x \: (Drama(x) \wedge Ballads(x)) \Rightarrow HE(x)\\

\: \: \: \: - CNF: \neg Drama(x) \vee \neg Ballads(x) \vee HE(x)\\

8. \forall x \: (Sone(x) \wedge Electro(x)) \Rightarrow DJH(x)\\

\: \: \: \: - CNF: \neg Sone(x) \vee \neg Electro(x) \vee DJH(x)\\

9. \forall x \: (Sone(x) \wedge Drama(x)) \Rightarrow SEO(x)\\

\: \: \: \: - CNF: \neg Sone(x) \vee \neg Drama(x) \vee SEO(x)\\

10. \forall x \: (Sone(x) \wedge Ballads(x)) \Rightarrow TAE(x)\\

\: \: \: \: - CNF: \neg Sone(x) \vee \neg Ballads(x) \vee TAE(x)\\

b) Using resolution, prove or disprove (by showing the complete process) that if a new user u1 is a fan of GG and identifies as Reveluv, then T AE will be a good recommendation.\\


c) Considering the same user u1, prove or disprove that HE will be a good recommendation.\\


d) Given what you know, if another user u2 claims to be a Sone, a Reveluv, a Blink, and likes Drama;
what are the possible artists and genre recommendations the system will provide? \\


\end{document}